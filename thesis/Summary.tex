\section*{Kurzfassung}
\vspace{0.5cm}

Diese Bachelorarbeit befasst sich mit dem Thema \glqq{Test-driven Development\grqq} (TDD) mit AngularJS und welches der beiden JavaScript Testing-Frameworks \glqq{Jasmine\grqq} und \glqq{Mocha\grqq} besser für diesen Zweck geeignet ist.

Die Themen \glqq{Testen im Web\grqq}, \glqq{Test-driven Development\grqq}, \glqq{AngularJS\grqq}, \glqq{Jasmine\grqq} und \glqq{Mocha\grqq} werden jeweils einzeln theoretisch erfasst und anhand von praktischen Beispielen erklärt. Anschließend wird eine AngularJS Applikation vorgestellt, welche jeweils mit den verschiedenen Testing-Frameworks test-driven entwickelt wurde. Anhand der Theorie und der entwickelten AngularJS Applikationen werden die Testing-Frameworks verglichen. Um eine vergleichbare Basis dafür zu schaffen, wurde das Framework Mocha um die Bibliotheken Chai und Sinon.JS erweitert.

Das Endergebnis der Analysen ergibt, dass AngularJS für TDD bereits gut geeignet ist, da AngularJS schon mit Test-Funktionalität (wie beispielsweise Mocks) ausgeliefert wird. Damit einhergehend ergibt sich, dass die Auswahl des richtigen Testing-Frameworks zweitranging wird.

Die Unterschiede der Frameworks sind kleiner als erwartet. Jasmine bedient sich eines Behaviour-driven Development (BDD)-Stils und Mocha lässt den Stil frei wählen (verfügbar sind BDD, TDD und ein \glqq{exports\grqq}-Interface). Jasmine ist dafür ein fertiges Framework, welches mit allen notwendigen Test-Features ausgeliefert wird (beispielsweise Spies und Mocks), während Mocha nicht über diese zusätzlichen, jedoch notwendigen Features verfügt und deshalb mit mehreren Bibliotheken erweitert werden muss.

\paragraph{Schlagwörter}
\textit{Test-Driven Development, TDD, AngularJS, Jasmine, Mocha}

% 201 Wörter
% 2 in Überschrift
