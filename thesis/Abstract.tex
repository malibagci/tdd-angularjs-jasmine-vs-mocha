\selectlanguage{english}
\subsection*{Abstract}
\vspace{0.5cm}

This bachelor thesis deals with the topics \glqq{Test-driven Development\grqq} (TDD) with AngularJS and which of the two testing frameworks \glqq{Jasmine\grqq} and \glqq{Mocha\grqq} is more suitable for this purpose.

Firstly, the topics \glqq{Testing in the web\grqq}, \glqq{Test-driven Development\grqq}, \glqq{AngularJS\grqq}, \glqq{Jasmine\grqq} and \glqq{Mocha\grqq} are reported theoretically and explained by examples. 
Secondly, an AngularJS application, which has been been developed test-driven (for each testing framework) is introduced. The testing frameworks are compared on basis of the theory and the developed AngularJS application. In order to make them comparable, the libraries Chai and Sinon.JS were added to the framework Mocha.

The final result of the analysis indicates that TDD, used in connection with AngularJS, works very well. This is because AngularJS is already shipped with a good test functionality (such as mocks). From this can be drawn that selecting the correct testing framework is of secondary importance.

The differences between the frameworks are minimal.
Jasmine uses a Behaviour-driven Development (BDD)-Style, whilst Mocha allows it to choose a style (available are BDD, TDD and an \glqq{exports\grqq}-interface). Jasmine, however, is a completed framework, which is shipped with all necessary features for testing (e.g. spies and mocks), while Mocha does not and therefore needs to be extended by several additional libraries.

\paragraph{Keywords:}
\textit{Test-Driven Development, TDD, AngularJS, Jasmine, Mocha}

\selectlanguage{ngerman}

% 214 Wörter
% 2 in Überschrift
